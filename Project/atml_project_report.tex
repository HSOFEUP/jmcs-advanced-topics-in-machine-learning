\documentclass[10pt,twocolumn,letterpaper]{article}

\usepackage{cvpr}
\usepackage{times}
\usepackage{epsfig}
\usepackage{graphicx}
\usepackage{amsmath}
\usepackage{amssymb}
\usepackage[backend=bibtex,
style=numeric,
bibencoding=ascii
%style=alphabetic
%style=reading
]{biblatex}
\addbibresource{bibliography.bib}

% Include other packages here, before hyperref.

% If you comment hyperref and then uncomment it, you should delete
% egpaper.aux before re-running latex.  (Or just hit 'q' on the first latex
% run, let it finish, and you should be clear).
\usepackage[breaklinks=true,bookmarks=false]{hyperref}

\cvprfinalcopy % *** Uncomment this line for the final submission

\def\cvprPaperID{****} % *** Enter the CVPR Paper ID here
\def\httilde{\mbox{\tt\raisebox{-.5ex}{\symbol{126}}}}

\begin{document}

%%%%%%%%% TITLE
\title{ATML Project Report: X-rays Bone age Prediction}

\author{Lukas Zbinden\\
University of Fribourg\\
ATML course, University of Bern, Spring 2018\\
{\tt\small lukas.zbinden@unifr.ch}
}

\maketitle
%\thispagestyle{empty}

%-------------------------------------------------------------------------
%%%%%%%%% ABSTRACT
\begin{abstract}
   Our project aimed at improving existing networks as well as exploring new ideas such as transfer learning for the prediction of the age based on hand X-rays (\cite{kaggleboneage}, \cite{stanfordboneage}).
\end{abstract}

%%%%%%%%% BODY TEXT
\section{Introduction}
In 2017 the Radiological Society of North America (RSNA) published a pediatric bone age prediction challenge (\cite{rsnacompetition}) that asked the ML community to develop an algorithm that most accurately determines skeletal age on a validation set of pediatric hand radiographs. The competition was won by \cite{16bitrsnachallenge} with a mean absolute difference (MAD) of 4.265 months. Following that, \cite{kaggleboneage} published the bone age data set on kaggle.com along with a strong baseline algorithm for further exploration and research. That is where we as group 3 stepped in to pick up this baseline with new ideas and unprecedented experiments.\\ We focused on three areas, namely the impact of different image preprocessing techniques, transfer learning with a different dataset and the use of different architectures, respectively, on the prediction task.

%-------------------------------------------------------------------------
\section{Related work}
We mainly focused on the work in \cite{kaggleboneage} and Kevin Mader, respectively, which we used as a baseline for our experiments. Further, we attempted to rebuild the competition winner's model according to \cite{16bitrsnachallenge} which we also used as a reference for our experiments. The techniques we applied in our work, such as data augmentation, in particular histogram equalization, as well as transfer learning are all based on contributions by the ML research community (e.g. \cite{1411.1792}, ...).

%------------------------------------------------------------------------
\section{Our methods}

\subsection{Image preprocessing}

\subsection{Challenging the 16bit competition winner}
One approach was to rebuild the winner's model of the 2017 RSNA ML challenge as close as possible according \cite{16bitrsnachallenge}.

\subsection{Transfer learning}
One approach was to use the large NIH chest X-rays data set with more than 112k images (\cite{nihchestxray}). 
To the best of our knowledge this had not been attempted before which was also confirmed by Kevin Mader.



%-------------------------------------------------------------------------
\section{Experiments}
Keras
GPU node03 server
values for hyperparameters

\subsection{Datasets}

\subsection{Image preprocessing}

\subsection{Hyperparameter tuning}
As is stated in \cite{1802.09596} ...

\subsection{Comparison to baseline}

\subsection{Analysis of results}

%------------------------------------------------------------------------
\section{Conclusions}

You must include your signed IEEE copyright release form when you submit
your finished paper. We MUST have this form before your paper can be
published in the proceedings.

Please direct any questions to the production editor in charge of these
proceedings at the IEEE Computer Society Press: Phone (714) 821-8380, or
Fax (714) 761-1784.

\printbibliography

\end{document}
